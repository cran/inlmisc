% Fonts
\usepackage{ifxetex}  % provides conditional for checking if XeTeX is running
\ifxetex
  \usepackage{fontspec}  % adds interface for advanced font selection
  \defaultfontfeatures{Ligatures=TeX}  % improves font behavior
\else
  \usepackage[T1]{fontenc}  % use Type 1 (PostScript) font encoding
\fi

\usepackage{mathptmx}  % define Times as default text font, and provide maths support
\usepackage{amsfonts}  % load American Mathematical Society fonts
\usepackage{textcomp}  % load Text Companion fonts

\renewcommand{\sfdefault}{lmss}  % use 'Latin Modern Sans Serif' for san-serif (sans) font family
\renewcommand{\ttdefault}{lmtt}  % use 'Latin Modern Sans Typewriter' for typewriter (mono) font family


% Layout
\usepackage[tmargin=1.000in, bmargin=0.667in, lmargin=0.750in, rmargin=0.583in]{geometry}  % define margins
\usepackage[pdftex]{lscape}  % place selected parts of a document in landscape


% Headers
\usepackage{fancyhdr}  % control page headers and footers
\pagestyle{fancy}
\renewcommand{\sectionmark}[1]{\markboth{#1}{}}
\renewcommand{\headrulewidth}{0pt}
\newcommand{\reportTitle}{Report Title}
\fancyhf{}
\fancyhead[LE]{\normalfont\bfseries\sffamily \thepage \quad \reportTitle}
\fancyhead[RO]{\normalfont\bfseries\sffamily \nouppercase{\leftmark} \quad \thepage}
\fancypagestyle{plain}{
  \fancyhf{}
  \fancyhead[LE]{\normalfont\bfseries\sffamily \thepage \quad \reportTitle}
  \fancyhead[RO]{\normalfont\bfseries\sffamily \thepage}
}


% Maths
\usepackage{amsmath}  % American Mathematical Society mathematical facilities
\usepackage{bm}  % access bold symbols in math mode
\usepackage{xfrac}  % typset fractions
\usepackage{siunitx}  % a comprehensive (SI) units package
\sisetup{
  input-ignore={,},
  input-decimal-markers={.},
  group-separator={,},
  group-minimum-digits=4,
  group-digits=integer,
  per-mode=symbol,
  separate-uncertainty=true,
  table-align-uncertainty=false
}
\DeclareSIUnit\curie{Ci}
\DeclareSIUnit\pCi{\pico\curie}


% Contents
\usepackage{titling}  % control over the typesetting of the \maketitle command
\pretitle{\begin{flushleft}\LARGE\bfseries\sffamily}
\posttitle{\end{flushleft}\vspace{2ex}}
\preauthor{\begin{flushleft}\large\sffamily}
\postauthor{\par\end{flushleft}}
\predate{}
\date{}
\postdate{}

\usepackage{tocloft}  % control 'Table of Contents', 'List of Figures', and 'List of Tables'
\renewcommand\cftsecfont{\sffamily}
\renewcommand\cftsubsecfont{\sffamily}
\renewcommand\cftsubsubsecfont{\sffamily}
\renewcommand\cfttoctitlefont{\Large\bfseries\sffamily}
\renewcommand\cftloftitlefont{\Large\bfseries\sffamily}
\renewcommand\cftlottitlefont{\Large\bfseries\sffamily}
\renewcommand{\cftdotsep}{0.5}
\renewcommand\cftsecleader{\cftdotfill{\cftdotsep}}
\renewcommand\cftsecpagefont{\normalfont}
\setlength{\cftfignumwidth}{2.55em}
\renewcommand\cftfigaftersnum{. }
\renewcommand\cfttabaftersnum{. }
\setlength{\cftbeforesecskip}{0.1cm}
\setlength{\cftbeforesubsecskip}{0.1cm}
\setlength{\cftbeforesubsubsecskip}{0.1cm}
\setlength{\cftbeforefigskip}{0.1cm}
\setlength{\cftbeforetabskip}{0.1cm}
\renewcommand*\listfigurename{Figures}
\renewcommand*\listtablename{Tables}


% Tables
\usepackage{array}  % extend the array and tabular environments
\usepackage{booktabs}  % enhance tables
\usepackage{makecell}  % tabular column heads and multilined cells
\usepackage{verbatimbox}  % deposit verbatim text in a box
\usepackage{multirow}  % tabular cells spanning multiple rows


% Captions
\usepackage{caption}  % customise captions in floating environments
\captionsetup{labelsep=period, labelfont=bf, justification=raggedright}
\captionsetup[figure]{skip=5pt}
\captionsetup[subfigure]{skip=-5pt, labelfont={bf,it}}

\usepackage{subcaption}  % add support for sub-captions
\renewcommand{\thesubfigure}{\Alph{subfigure}}  % identify sub-figures with letter in sub-caption

\setcounter{secnumdepth}{0}  % allow only \section in Table of Contents


% References
\usepackage[backend=bibtex, style=authoryear-comp, maxbibnames=99, maxcitenames=3, natbib=true, dashed=false]{biblatex}
\setcounter{biburlnumpenalty}{100}
\setcounter{biburlucpenalty}{100}
\setcounter{biburllcpenalty}{100}
\renewcommand\refname{References Cited}
\DefineBibliographyStrings{english}{andothers={and~others}}
\setlength{\bibitemsep}{\baselineskip}
\DeclareFieldFormat{title}{#1}
\DeclareFieldFormat[article,thesis,report,incollection,inproceedings,misc]{title}{#1}
\DeclareFieldFormat[incollection,inproceedings]{booktitle}{#1}
\DeclareNameAlias{sortname}{family-given}
\DeclareNameAlias{default}{family-given}
\DefineBibliographyStrings{english}{pagetotal={p\adddot}, pagetotals={p\adddot}}
\DefineBibliographyStrings{english}{references={References Cited}}
\renewcommand*{\newunitpunct}{\addcomma\space}
\renewcommand*{\intitlepunct}{\space}
\DeclareFieldFormat[article]{volume}{v.\addnbspace #1}
\DeclareFieldFormat[article]{number}{no.\addnbspace #1}
\DeclareFieldFormat[article]{journaltitle}{#1}
\renewbibmacro*{volume+number+eid}{%
  \printfield{volume}%
  \setunit{\addcomma\space}%
  \printfield{number}%
  \setunit{\addcomma\space}%
  \printfield{eid}}
\renewbibmacro*{journal+issuetitle}{%
  \usebibmacro{journal}%
  \setunit*{\addcomma\space}%
  \iffieldundef{series}
    {}
    {\newunit
     \printfield{series}%
     \setunit{\addcomma\space}}%
  \usebibmacro{volume+number+eid}%
  \setunit{\addspace}%
  \usebibmacro{issue+date}%
  \setunit{\addcolon\space}%
  \usebibmacro{issue}%
  \newunit}

\usepackage{xpatch}  % macro patching commands
\xpatchbibmacro{date+extradate}{%
  \printtext[parens]%
}{%
  \setunit*{\addcomma\space}%
  \printtext%
}{}{}


% Sectioning
\usepackage{sectsty}  % control sectional headers
\sectionfont{\Large\bfseries\sffamily}
\subsectionfont{\large\bfseries\sffamily}
\subsubsectionfont{\large\mdseries\sffamily}

\widowpenalties 1 10000 % penalties that control line or page breaks

\makeatletter  % align floating objects (e.g. figures) along the top of a page
\setlength{\@fptop}{0pt}
\setlength{\@fpbot}{0pt plus 1fil}
\makeatother

\usepackage{indentfirst}  % indent first paragraph after section header

\usepackage{ragged2e}  % control ragged text
\setlength\RaggedRightParindent{15pt}  % set paragraph indentation

\raggedbottom  % make all pages the height of the text on that page

\usepackage{float}  % provides the H float modifier option


% Lists
\usepackage{enumitem}  % control layout of itemize, enumerate, description
\renewcommand*\descriptionlabel[1]{\hspace\leftmargin$#1$}  % make descriptions appear on left-hand side with an indent


% Miscellaneous
\newcommand{\R}{{\normalfont\textsf{R}}{}}  % set format for R-programming character

\renewenvironment{knitrout}{\setlength{\topsep}{0mm}}{}  % remove blank lines around code blocks

\extrafloats{1000}  % allocate more floats

\usepackage[multiple]{footmisc}  % typesetting of footnotes

\usepackage[]{color}  % control colors
\definecolor{Red}{rgb}{0.7,0,0}
\definecolor{Blue}{rgb}{0,0,0.8}

\pretolerance=5000  % reduce hyphenation
\tolerance=9000  % influences paragraph breaking
\emergencystretch=0pt  % prevent writing into margin


% Links
\usepackage[pdfa, hidelinks, pdflang={en-US}]{hyperref}  % support for hyperlinks
\hypersetup{
  colorlinks,
  urlcolor=Red,
  linkcolor=Blue,
  citecolor=Blue,
  linktoc=page
}
